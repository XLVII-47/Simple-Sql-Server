\documentclass[15pt]{scrartcl} % Font size
\usepackage{listings}
\newcommand{\horrule}[1]{\noindent\rule{\linewidth}{#1}}

\title{	
	\normalfont\normalsize
    \vspace{150pt}
	\textsc{\Large Gebze Technical University}\\ % Your university, school and/or department name(s)
	\vspace{100pt} 
	\textsc{\Large CSE 344 System Programming Final Report}\\ % The assignment title
	\vspace{12pt} 
}

\author{\huge Muhammed Mücahit ÜÇER \\ 161044030} % Your name



\begin{document}

\maketitle % Print the title

\newpage
\section{\huge Server}

\subsection{struct Serversetup}
1- All information about the server is kept in this structure.\\\\
2- As many threads and sockets are created as the poolsize is.\\\\
3- After the sockets are created, they are initialized as EMPTY.\\\\
4-Scheduler() function waits for a connection, when it connects, it searches for an EMPTY socket and assigns the created socketfd to it, then broadcasts to activate the threads.\\\\
5-All threads in await state start running and look for the socket assigned to them, if they are assigned, they initiate communication with the client.\\\\
6- Before table records are sent, some serialize operations are applied and sent with the send str() function.

\section{\huge Csv}
\subsection{struct Csv}
1- After reading the dataset, all information is saved in this structure.\\\\
2- It also contains the necessary variables to implement the reader-writer paradigm.\\\\
3- After the data is read from the file and parsed, it is stored in a 2d string array.\\\\
4- The reader-writer paradigm was implemented as we saw in the lesson.\\\\
5- The reason why I use 2d string array is completely easy to implement.
\subsection{readerLock readerUnlock sample}

\begin{lstlisting}[
language=C,
frame=single,
showstringspaces=false,
numbers=left,
numberstyle=\tiny,
]
void reader_lock(struct Csv *handle){
    if (pthread_mutex_lock(&handle->m) != 0){
     errexit("reader_lock, pthread_mutex_lock");
    }
    while ((handle->AW + handle->WW) > 0){
     handle->WR++;
     pthread_cond_wait(&handle->okToRead, &handle->m);
     handle->WR--;
    }
    handle->AR++;
     if (pthread_mutex_unlock(&handle->m) != 0){
     errexit("reader_lock, pthread_mutex_unlock");
    }
}
\end{lstlisting}

\begin{lstlisting}[
language=C,
frame=single,
showstringspaces=false,
numbers=left,
numberstyle=\tiny,
]
void reader_unlock(struct Csv *handle)
{
    if (pthread_mutex_lock(&handle->m) != 0){
     errexit("reader_unlock, pthread_mutex_lock");
    }
    handle->AR--;
    if (handle->AR == 0 && handle->WW > 0){
     if (pthread_cond_signal(&handle->okToWrite) != 0){
        errexit("reader_unlock, pthread_cond_signal");
       }
    }
    if (pthread_mutex_unlock(&handle->m) != 0){
     errexit("reader_unlock, pthread_mutex_unlock");
    }
}
\end{lstlisting}

It is used in the searchforquery function to wrap read or write operations.

\section{\huge Client}

It reads the queries from the file and sends the query to the server if it matches its own id. It does this in a loop. If there is more than one query, it sends them all over the same connection.

\end{document}
